<div>

This picture shows Boy's surface, which is a projection of the real 
projective plane $\mathbb{R}P^2$ into the three dimensional space 
$\mathbb{R}^3$.

<br/><br/>

Remember that the trefoil knot (for example) is really a curve in three
dimensions that does not intersect itself.  When we draw a two-dimensional
projection of the trefoil we get a curve that crosses over itself, but 
this is just an artefact; there is not enough space in the plane to draw 
the trefoil properly.  In the same way, $\mathbb{R}P^2$ does not really 
intersect itself but it can only be represented without self-intersections 
in four dimensions or more.  Boy's surface is an imperfect representation 
of $\mathbb{R}P^2$ in three dimensions.

<br/><br/>

A simpler way to represent $\mathbb{R}P^2$ is to paste the sides of a square, 
like this:

<br/><br/>

<div style="width:600px; margin-left: auto; margin-right: auto;">
<img src="boys_net.png"/>
</div>

<br/><br/>

Here is a model of Boy's surface at the famous mathematical research centre 
at Oberwolfach in southern Germany:

<br/><br/>

<img width="600" src="boys.jpg"/>


</div>