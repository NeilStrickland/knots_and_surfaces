\documentclass{amsart}
\usepackage{hyperref}
\usepackage{fullpage}
\usepackage{amsrefs}

\usepackage[matrix,arrow]{xy}
\usepackage{xypdf}
\newdir{ >}{{}*!/-9pt/\dir{>}}

\newcommand{\Z}         {{\mathbb{Z}}}
\newcommand{\R}         {{\mathbb{R}}}
\newcommand{\C}         {{\mathbb{C}}}

\newcommand{\sm}        {\setminus}
\newcommand{\st}        {\;|\;}
\newcommand{\tm}        {\times}

\newcommand{\Sg}        {\Sigma}
\newcommand{\Dl}        {\Delta}
\newcommand{\Lm}        {\Lambda}

\newcommand{\al}        {\alpha}
\newcommand{\bt}        {\beta}
\newcommand{\gm}        {\gamma}
\newcommand{\dl}        {\delta}
\newcommand{\ep}        {\epsilon}
\newcommand{\tht}       {\theta}
\newcommand{\kp}        {\kappa}
\newcommand{\lm}        {\lambda}
\newcommand{\om}        {\omega}
\newcommand{\sg}        {\sigma}
\newcommand{\zt}        {\zeta}

\newcommand{\tU}	{\widetilde{U}}
\newcommand{\tV}	{\widetilde{V}}
\newcommand{\tW}	{\widetilde{W}}

\newcommand{\tf}	{\widetilde{f}}
\newcommand{\tg}	{\widetilde{g}}
\renewcommand{\th}        {\widetilde{h}}

\newcommand{\ip}[1]     {\langle #1\rangle}
\newcommand{\sse}       {\subseteq}

\renewcommand{\:}{\colon}

\newtheorem{theorem}{Theorem}[section]
\newtheorem{conjecture}[theorem]{Conjecture}
\newtheorem{lemma}[theorem]{Lemma}
\newtheorem{proposition}[theorem]{Proposition}
\newtheorem{corollary}[theorem]{Corollary}
\theoremstyle{definition}
\newtheorem{remark}[theorem]{Remark}
\newtheorem{definition}[theorem]{Definition}
\newtheorem{example}[theorem]{Example}
\newtheorem{construction}[theorem]{Construction}

\newtheorem{notation}{Notation}
\renewcommand{\thenotation}{} % make the notation environment unnumbered

%\numberwithin{equation}{subsection}

\begin{document}
\title{Combinatorial structures for knot theory}
\author{N.~P.~Strickland}

\maketitle 

\section{Basic definitions}

\begin{definition}
 We let $\Lm$ denote the group freely generated by elements $\rho$ and
 $\chi$ subject only to relations $\rho^4=\chi^2=1$.  We put
 $\lm=\rho^{-1}\chi$ and $\pi=\rho^2\chi$.  We define
 $c\:\Lm\to\{\pm 1\}$ by $c(\rho)=c(\chi)=-1$ (so $\ker(c)$ is the
 smallest normal subgroup containing $\lm$).  
\end{definition}

\begin{definition}\label{defn-universe}
 A \emph{combinatorial link universe} (or just \emph{universe}) is a
 finite set $A$ with an action of $\Lm$ such that 
 \begin{itemize}
  \item[(a)] The subgroups $\ip{\chi}\simeq C_2$ and $\ip{\rho}\simeq C_4$
   both act freely on $A$.
  \item[(b)] For each $a\in A$, the stabiliser group is contained in
   $\ker(c)$. 
 \end{itemize}
\end{definition}

\begin{remark}\label{rem-rho-squared}
 It actually follows from~(b) that $\chi$ has no fixed points, so half
 of axiom~(a) is redundant.  It also follows from~(b) that $\rho$ can
 have no fixed points, but $\rho^2$ could have fixed points.  Thus, we
 could replace axiom~(a) by the assumption that $\rho^2$ has no fixed
 points. 
\end{remark}

\begin{example}\label{eg-loop}
 The smallest nonempty example is the set $\Z/4$ with $\rho(i)=1+i$
 and $\chi(i)=1-i$.  This is called a \emph{loop}.
\end{example}

\begin{remark}
 Suppose we start with a geometric link universe $U_0\subset\R^2$,
 with crossing set $X_0$.  Suppose also that there are no isolated
 circles.  A \emph{directed arc} for $U_0$ is pair $(u,m)$, where $u$
 a connected component of $U_0\sm X_0$, and $m$ is an orientation of
 $u$.  We define $A$ to be the set of directed arcs.  We define
 $\chi\:A\to A$ by $\chi(u,m)=(u,-m)$.  Next, we can use $m$ to define
 the starting point $x=\sg(u,m)\in X_0$ of $(u,m)$.  Moving
 anticlockwise around $x$, let $v$ be the next component of
 $U_0\sm X_0$ that we encounter, and let $n$ be the unique orientation
 of $v$ such that the part of $v$ that we encounter is the starting
 end of $v$ relative to $n$.  We then put $\rho(u,m)=(v,n)$.  This
 gives a combinatorial link universe.  To see that axiom~(b) is
 satisfied, note that the sum of the arcs defines a mod two homology
 class $u$ on $U_0$.  For $x\in\R^2\sm U_0$, define $\mu(x)=1$ if $u$
 becomes trivial in $H_1(\R^2\sm\{x\};\Z/2)$, and $\mu(x)=-1$
 otherwise.  Then define $\tht\:A\to\{\pm 1\}$ by taking $\tht(a)$ to
 be the value of $\mu$ on the left hand side of $a$ (with respect to
 the given orientation).  We find that $\tht\rho=\tht\chi=-\tht$, so
 $\tht\al=c(\al)\tht$ for all $\al\in\Lm$, and axiom~(b) follows from
 this.

 If we want to handle a geometric universe that contains isolated
 circles, we can just add a small twist to each such circle.  
\end{remark}

\begin{definition}
 Given a universe $A$:
 \begin{itemize}
  \item The \emph{blocks} are the $\Lm$-orbits in $A$.  Each block
   can be considered as a universe.
  \item The \emph{components} are the $\ip{\chi,\rho^2}$-orbits in $A$.
  \item The \emph{crossings} are the $\ip{\rho}$-orbits in $A$.
  \item The \emph{edges} are the $\ip{\chi}$-orbits in $A$.
  \item The \emph{faces} are the $\ip{\lm}$-orbits in $A$.
  \item A \emph{chequer function} is a map $\tht\:A\to\{\pm 1\}$ with
   $\tht\chi=\tht\rho=-\tht$.
  \item A \emph{height function} is a map $\om\:A\to\{\pm 1\}$ with
   $\om\rho=-\om$.
  \item An \emph{orientation} is a map $\dl\:A\to\{\pm 1\}$ with
   $\dl\chi=\dl\rho^2=-\dl$.
 \end{itemize}
\end{definition}

\begin{remark}
 If a universe has $n$ crossings, we see that it has $4n$ arcs and
 $2n$ edges.  
\end{remark}

\begin{remark}
 Axiom~(b) in Definition~\ref{defn-universe} is equivalent to the
 statement that each block admits a chequer function, and thus that
 the whole universe admits a chequer function.
\end{remark}

\begin{remark}\label{rem-dihedral}
 Let $\Dl$ be the subgroup of $\Lm$ generated by $\rho^2$ and $\chi$,
 so the element $\pi=\rho^2\chi$ lies in $\Dl$.  Note that
 $\chi\pi=\pi^{-1}\chi$ and $\rho^2\pi=\pi^{-1}\rho^2$.  Every element
 of $\Dl$ can be written as $\pi^m$ or $\pi^m\chi$ for some $m\in\Z$.
 The elements $\pi^m$ have infinite order, but the elements
 $\pi^m\chi$ are involutions.  We have
 $\pi^{2k}\chi=\pi^k\chi\pi^{-k}$, but
 $\pi^{2k+1}\chi=\pi^k\rho^2\pi^{-k}$, so every involution is
 conjugate to $\pi$ or $\rho^2$.

 Note that $\Dl$ acts on $A$ in such a way that $\rho^2$ and $\chi$
 are free involutions.  As all other involutions in $\Dl$ are
 conjugate to $\chi$ or $\rho^2$, they also act freely.  Thus, the
 stabiliser of each point is of the form $\ip{\pi^m}$ for some $m$.
 Moreover, we have $c(\pi)=-1$ so $m$ must be even, say $m=2k$.  Note
 also that $\ip{\pi^{2k}}$ is normal in $\Dl$, so we get a free action
 of $\Dl/\ip{\pi^{2k}}$ on the orbit.  In particular, the size of each
 orbit is divisible by $4$.
\end{remark}

\begin{definition}
 Let $A$ be a universe, and let $X=A/\ip{\rho}$ be the set of
 crossings.  Let $\sg\:A\to X$ be the natural projection, and put
 $\tau=\sg\chi$.  If $a\in A$ with $\sg(a)=x$ and $\tau(a)=y$, we will
 say that $a$ is an arc from $x$ to $y$.
\end{definition}

\begin{definition}
 The \emph{mirror image} of $A$ is the same set with the same $\chi$,
 but with $\rho$ replaced by $\rho^{-1}$.  
\end{definition}

\begin{remark}
 Note that the map $\lm=\rho^{-1}\chi$ is conjuagate (via $\chi$) to
 the inverse of $\rho\chi$.  This means that the
 $\ip{\rho^{-1}\chi}$-orbits biject with the $\ip{\rho\chi}$-orbits,
 so a universe and its mirror image have the same number of faces.
\end{remark}

\textbf{To do:}
\begin{itemize}
 \item Basic examples: twist knots, torus knots, closures of braids,
  chains of links.
\end{itemize}

\section{Geometric realisation}

\begin{definition}
 The \emph{geometric realisation} of $A$ is 
 \[ |A| = ((A\tm\Dl_2)/\sim), \]
 where we take $\Dl_2=\{(x,y,z)\in [0,1]^2\st x+y+z=1\}$, and the
 equivalence relation is generated by $(a;x,y,0)\sim(\chi(a),y,x,0)$
 and $(a;0,x,z)\sim(\lm(a);x,0,z)$.
\end{definition}

\begin{proposition}
 The space $|A|$ is a closed surface, and the set of chequer functions
 bijects naturally with the set of orientations.  There is a natural
 CW structure, with a two-cell for each face, a one-cell for each
 edge, and a zero-cell for each crossing.  Thus, if there are $m$
 faces and $n$ crossings then the Euler characteristic is $m-n$.
\end{proposition}
\begin{proof}
 Put $P=A\tm\Dl_2$, and let $q\:P\to |A|$ be the natural projection
 map.  Note that $P$ is compact Hausdorff, and our equivalence
 relation is easily seen to give a closed subspace of $P\tm P$, so the
 quotient space $|A|$ is again compact Hausdorff.
 We say that a subset $T\sse P$ is \emph{saturated} if
 $T=q^{-1}(q(T))$, or equivalently $t\in T$ whenever $t\sim t'$ for
 some $t'\in T$.  We will cover $P$ by saturated open sets as defined
 below. 

 Fix an arc $a\in A$.
 \begin{itemize}
  \item We put 
   \[ \tU_a=\{(\rho^k(a);x,y,z)\in P\st k\in\Z/4,\;x>1/2\}
            \amalg\{(\chi\rho^k(a);x,y,z)\in P\st k\in\Z/4,\;y>1/2\}.
   \]
   Note here that the elements $\rho^i(a)$ are all distinct, and the
   elements $\chi\rho^k(a)$ are all distinct, but it can happen in
   exceptional cases that $\chi\rho^k(a)=\rho^m(a)$.  Nonetheless, the
   conditions on $x$ and $y$ ensure that the above union is disjoint
   even in those cases.
  \item We put
   \[ \tV_a = \{(\chi^j(a);x,y,z)\in P\st j\in\Z/2,\;xy>0\}. \]
  \item Now let $d$ be the smallest positive integer such that
   $\lm^d(a)=a$, and put
   \[ \tW_a=\{(\lm^k(a);x,y,z)\in P\st k\in\Z/d,\;z>0\}. \]
 \end{itemize}
 It is straightforward to check that each of these sets is open in $P$
 and is saturated, so the images $U_a=q(\tU_a)$, $V_a=q(\tV_a)$ and
 $W_a=q(\tW_a)$ are open in $|A|$.

 We now define maps $\tf_a$, $\tg_a$ and $\th_a$ from $\tU_a$, $\tV_a$
 and $\tW_a$ to $\C$.  These will use the map
 $\phi_{\al\bt}\:\Dl_2\to\C$ given by 
 \[ \phi_{\al\bt}(x,y,z) =
     (x+y)\exp\left(2\pi i\frac{x\al+y\bt}{x+y}\right).
 \]
 It is straightforward to check that this is continuous even at
 $(0,0,1)$, and if $\bt-\al<2\pi$ it gives a homeomorphism from
 $\Dl_2$ to $\{re^{2\pi i t}\st 0\leq r\leq 1,\;\al\leq t\leq\bt\}$.  
 We put 
 \begin{align*}
  \tf_a(\rho^k(a);x,y,z) &= 2\phi_{2k/8,(2k+1)/8}(y,z,x) &
  \tf_a(\chi\rho^k(a);x,y,z) &= 2\phi_{2k/8,(2k-1)/8}(x,z,y) \\
  \tg_a(a;x,y,z) &= x-y+z\sqrt{1-(x-y)^2}i &
  \tg_a(\chi(a);x,y,z) &= y-x-z\sqrt{1-(x-y)^2}i \\
  \th_a(\lm^k(a);x,y,z) &= \phi_{k/d,(k+1)/d}(x,y,z).
 \end{align*}
 Note that $\tf_a$ sends the equivalent points $(\rho^k(a);x,y,0)$ and
 $(\chi\rho^k(a);y,x,0)$ to $2i^ky$, and it also sends the
 equivalent points $(\rho^k(a);x,0,z)$ and
 $(\lm^{-1}\rho^k(a);0,x,z)=(\chi\rho^{k+1}(a);0,x,z)$ to
 $2i^{k+1/2}z$.  These are the only relations that occur in
 $\tU_a$, so there is an induced map $f_a\:U_a\to\C$.  Similarly,
 $\tg_a$ respects the relation $(a;x,y,0)\sim(\chi(a);y,x,0)$ and so
 induces a map $g_a\:V_a\to\C$.  Moreover, $\th_a$ respects the
 relations $(\lm^k(a);0,x,z)\sim(\lm^{k+1}(a);x,0,z)$ and so induces
 $h_a\:W_a\to\C$.  It is not hard to see that in each case, we get a
 homeomorphism with the open unit disc.  

 All claims in the proposition are now reasonably clear.
\end{proof}

\begin{definition}
 We say that $A$ is \emph{spherical} if it admits a chequer function,
 and for each block $B$, the space $|B|$ has euler characteristic two
 (and so is homeomorphic to $S^2$).
\end{definition}

\section{Reidemeister moves}

\begin{definition}
 Let $A$ be a universe, and let $a\in A$ be an arc with
 $\lm(a)=a$ (so $\chi(a)=\rho(a)$) but
 $\chi\rho^2(a)\neq\rho^3(a)$.  We define  
 \[ A^* = A \sm \{\rho^k(a)\st k\in\Z/4\}, \]
 and we let $\rho^*$ denote the restriction of $\rho$ to $A^*$.  Note
 that $\chi$ exchanges $a$ and $\rho(a)$, so it preserves
 $A\sm\{a,\rho(a)\}=A^*\amalg\{\rho^2(a),\rho^3(a)\}$.  By assumption,
 $\chi$ does not exchange $\rho^2(a)$ and $\rho^3(a)$.  It follows
 that the arcs $b=\chi\rho^2(a)$ and $c=\chi\rho^3(a)$ lie in $A^*$
 and are distinct, and that $\chi$ preserves $A^*\sm\{b,c\}$.  We
 define $\chi^*$ to be the unique involution on $A^*$ that exchanges
 $b$ and $c$ but otherwise agrees with $\chi$.  Note that if $\tht$ is
 a chequer function on $A$ then $\tht(b)+\tht(c)=0$ so $\tht|_{A^*}$
 is a chequer function on $A^*$.  Thus, $A^*$ is again a universe.  We
 define $R_1(A,a)=A^*$, and we call this operation \emph{Reidemeister
  move 1}.
\end{definition}

\begin{remark}
 The condition $\chi\rho^2(a)\neq\rho^3(a)$ means that we do not allow
 a Reidemeister move that would convert a figure eight to an untwisted
 loop.  
\end{remark}

\begin{definition}
 Let $A$ be a universe, and let $a\in A$ be an arc with
 $\lm^2(a)=a$, or equivalently $\chi\rho(a)=\rho^3\chi(a)$.  This
 means that among the arcs $b_{ijk}=\chi^i\rho^j\chi^k(a)$ we have the
 coincidences $b_{110}=b_{031}$ and $b_{131}=b_{010}$, as well as the
 automatic identities $b_{101}=b_{000}$ and $b_{100}=b_{001}$.
 Suppose that there are no further coincidences, so we have $12$
 distinct arcs of the form $b_{ijk}$.  We put 
 \[ A^* = A \sm \{b_{0jk}\st j\in\Z/4,\;k\in\Z/2\}, \]
 and note that this is preserved by $\rho$.  We define $\rho^*$ to be
 the restriction of $\rho$ to $A^*$.  Next, we observe that the set
 \[ A'=A\sm\{b_{ijk}\st i,k\in\Z/2,\;j\in\Z/4\} \]
 is preserved by $\chi$, and that  $A^*$ consists of $A'$ together
 with the four distinct points $b_{111}$, $b_{120}$, $b_{121}$ and
 $b_{130}$.  We let $\chi^*$ denote the unique involution on $A^*$
 that agrees with $\chi$ on $A'$ and satisfies
 $\chi^*(b_{111})=b_{130}$ and $\chi^*(b_{120})=b_{121}$.  One can
 check that any chequer function on $A$ restricts to give a chequer
 function on $A^*$, so $A^*$ is again a universe.  We define
 $R_2(A,a)=A^*$, and we call this operation \emph{Reidemeister move
  2}. 
\end{definition}

\begin{remark}
 Our auxiliary condition forbids all cases in which there are some
 monogons involved, but that should be harmless, because we can create
 or destroy monogons using Reidemeister move 1.  Our auxiliary
 condition also forbids cases where Reidemeister move 2 would create a
 floating circle.  This should again be harmless, because we can first
 add a twist to the circle, and then remove it.
\end{remark}

\begin{definition}
 Let $A$ be a universe, and let $a\in A$ be an arc such that
 $\lm^3(a)=a$.  We will define a new universe $A^*$, with the same
 underlying set as $A$ and the same action of $\chi$, but with a
 different map $\rho^*$ in place of $\rho$.  To define this, let $B$
 be the subset of $A$ consisting of the elements
 $b_{ijk}=\rho^{2i}\chi^j\lm^k(a)$, for $i,j\in\Z/2$ and $k\in\Z/3$.
 We will assume that there are no coincidences among these elements,
 so $|B|=12$.  Using $\rho=\chi\lm^{-1}$ we see that
 \begin{align*}
  \rho(b_{00i}) &= b_{0,1,i-1} & \rho(b_{01i}) &= b_{1,0,i+1} \\
  \rho(b_{10i}) &= b_{1,1,i-1} & \rho(b_{11i}) &= b_{0,0,i+1}.
 \end{align*}
 Thus, the set $B$ is closed under $\rho$.  We define $\rho^*$ to be
 the same as $\rho$ on $A\sm B$, and we put 
 \begin{align*}
  \rho^*(b_{00i}) &= b_{1,1,i+1} & \rho^*(b_{01i}) &= b_{0,0,i-1} \\
  \rho^*(b_{10i}) &= b_{0,1,i+1} & \rho^*(b_{11i}) &= b_{1,0,i-1}.
 \end{align*}
 One can check that $(\rho^*)^2(b_{ijk})=b_{i+2,j,k}=\rho^2(b_{ijk})$.
 As the map $(\rho^*)^2=\rho^2$ is a free involution, it follow that
 $\rho^*$ gives a free action of $C_4$ on $A^*$.  If $\tht$ is any
 chequer function for $A$, we see that $\tht(b_{ijk})=(-1)^j\tht(a)$,
 and it follows that $\tht\rho^*=-\tht$, so $\tht$ is also a chequer
 function for $A^*$.  Thus, $A^*$ is again a universe.  We define
 $R_2(A,a)=A^*$, and we call this operation
 \emph{Reidemeister move 3}.
\end{definition}

\begin{remark}
 The auxiliary condition $|B|=12$ forbids the case where $\lm(a)=a$.
 I suspect that all other kinds of coincidences are automatically
 excluded by the definition of a universe, but I have not checked that
 completely. 
\end{remark}

\begin{definition}
 Universes $A$ and $A'$ are \emph{Reidemeister equivalent} if they are
 related by a series of Reidemeister moves.  More explicitly, there
 should exist a sequence $B_0,\dotsc,B_r$ of universes where
 \begin{itemize}
  \item $B_0\simeq A$ and $B_r\simeq A'$;
  \item For all $i$, either $B_{i+1}$ is obtained (up to isomorphism)
   by applying a Reidemeister move to $B_i$, or $B_i$ is obtained by
   applying a Reidemeister move to $B_{i+1}$.
 \end{itemize}
\end{definition}

\textbf{To do:}
\begin{itemize}
 \item Effect of Reidemeister moves on crossings and orientations.
\end{itemize}

\section{The Jones polynomial}

\begin{definition}
 \begin{itemize}
  \item A \emph{link diagram} is a universe equipped with a height
   function $\om\:A\to\{\pm 1\}$.  
  \item An \emph{oriented universe} is a universe equipped with an
   orientation $\dl\:A\to\{\pm 1\}$.  In this context we put
   $A_+=\dl^{-1}\{1\}$ and $A_-=\chi(A_+)=\dl^{-1}\{-1\}$.
  \item An \emph{oriented link diagram} is a universe equipped with
   both a height function and an orientation.
 \end{itemize}
\end{definition}

\begin{lemma}
 Let $A$ be an oriented universe.  Then there is a
 unique map $\zt\:X\to A$ with $\sg\zt=1$ and 
 \[ \dl\zt(x) = \dl\rho\zt(x)=1,\hspace{4em}
    \dl\rho^2\zt(x) = \dl\rho^3\zt(x)=-1.
 \]
\end{lemma}

\begin{definition}
 Let $D$ be an oriented link diagram.  The \emph{sign} of a crossing
 $x\in X$ is $\om\zt(x)$.  We also define the \emph{writhe}
 $w(D)=\sum_{x\in X}\om\zt(x)$.
\end{definition}

\begin{definition}
 Let $A$ be a link universe.  A \emph{state} of $A$ is a free
 involution $\xi\:A\to A$, such that for all $a$ we have
 $\sg(a)\in\{\rho(a),\rho^{-1}(a)\}$.  We write $\Sg(A)$ for the set
 of states.  Given a state $\xi$, we put $\kp_0(\xi)=A/\ip{\chi,\xi}$,
 and call this the set of \emph{cut components} for $\xi$.  We put
 $d(\xi)=|\kp_0(\xi)|$.
\end{definition}

\begin{remark}
 Fix a directed arc $a_0$, and put $x=\sg(a_0)$ and $a_i=\rho^i(a_0)$,
 so $\sg^{-1}\{x\}=\{a_0,a_1,a_2,a_3\}$.  Any state $\xi$ must act on
 this set as a transposition pair, but it cannot exchange $a_0$ and
 $a_2$.  Thus, there are precisely two possibilities: the restriction
 of $\xi$ is either $(a_0\;a_1)(a_2\;a_3)$ or $(a_0\;a_3)(a_1\;a_2)$.
 Thus, we have $|\Sg(A)|=2^{|X|}$.
\end{remark}

\begin{definition}
 Now suppose we have a link diagram $D$ with height function $\om$.
 Given a state $\xi$ and a crossing $x\in X$, define
 $\phi(\xi,x)\in\{A,B\}$ as follows: 
 \begin{itemize}
  \item If $\xi(a)=\rho^{-\om(a)}(a)$ for all $a\in\sg^{-1}\{x\}$,
   then $\phi(\xi,x)=A$.
  \item If $\xi(a)=\rho^{\om(a)}(a)$ for all $a\in\sg^{-1}\{x\}$,
   then $\phi(\xi,x)=B$.
 \end{itemize}
 (One can check that these are the only two possibilities.)

 We also define 
 \begin{align*}
  \phi(\xi) &= \prod_{x\in X}\phi(\xi,x)\in\Z[A,B] \\
  \psi(D) &= \sum_{\xi} \phi(\xi) C^{d(\xi)}\in\Z[A,B,C].
 \end{align*}

 We then let $\psi_0(D)$ denote the image of $\psi(D)$ in the ring 
 \[ \Z[A,B,C]/(AB-1,ABC+A^2+B^2) = \Z[A^{\pm 1}]. \]

 Finally, if $D$ also has an orientation we put
 $\psi_1(D)=(-A^{-3})^{w(D)}\psi_0(D)$.  We call $\psi_1(D)$ the
 \emph{Jones polynomial} of $D$.  
\end{definition}

\begin{remark}
 It is traditional to normalise the Jones polynomial so that the Jones
 polynomial of an unlinked circle is $1$; this has the effect that the
 Jones polynomial converts connected sums (in a sense that we have
 not defined here) to products.  We have instead normalised the Jones
 polynomial so that $\psi_1(\emptyset)=1$ and
 $\psi_1(D\amalg D')=\psi_1(D)\psi_1(D')$.
\end{remark}

\begin{theorem}
 If $D$ and $D'$ are Reidemeister equivalent then
 $\psi_1(D)=\psi_1(D')$. 
\end{theorem}

\textbf{To do:}
\begin{itemize}
 \item Discuss the skein relation.
\end{itemize}

\section{Dowker-Thistlethwaite theory}

\begin{definition}\label{defn-dowker-zero}
 Recall that $\Dl=\ip{\chi,\rho^2}=\ip{\chi,\pi}$.  For $n>0$ we
 define $DT_0(n)=\Z/(2n)\tm\{\pm 1\}$, with $\Dl$-action by
 $\chi(i,\ep)=(i,-\ep)$ and $\pi(i,\ep)=(i+\ep,\ep)$ (so
 $\rho^2(i,\ep)=(i-\ep,-\ep)$).  
\end{definition}

\begin{definition}\label{defn-dowker}
 Let $\sg$ be an involution on $\Z/(2n)$ such that
 $\sg(i)=1-i\pmod{2}$ for all $i$, and let $\phi\:\Z/(2n)\to\{\pm 1\}$
 be a map such that $\phi\sg=-\phi$.  We define $DT(\sg,\phi)$ to be
 the set $\Z/(2n)\tm\{\pm 1\}$ equipped with the maps
 $\chi(i,\ep)=(i,-\ep)$ and 
 \begin{align*}
  \rho(i, 1) &= \begin{cases}
                 (\sg(i),1)    & \text{ if } \phi(i)=1 \\
                 (\sg(i)-1,-1) & \text{ if } \phi(i)=-1 
                \end{cases} \\
  \rho(i,-1) &= \begin{cases}
                 (\sg(i+1)-1,-1) & \text{ if } \phi(i+1)=1 \\
                 (\sg(i+1), 1)   & \text{ if } \phi(i+1)=-1.
                \end{cases} \\
 \end{align*}
\end{definition}

\begin{proposition}
 $D(\sg,\phi)$ is a universe, with underlying $\Dl$-set $DT_0(n)$. 
\end{proposition}
\begin{proof}
 It is clear that $\chi$ acts on $DT(\sg,\phi)$ in the same way that
 it acts on $DT_0(n)$.  A check of cases shows that $\rho^2$ also acts
 on $DT(\sg,\phi)$ in the same way that it acts on $DT_0(n)$ (and the
 same then follows for $\pi=\rho^2\chi$).  It is easy to see that
 $\rho^2$ has no fixed points, and that the map
 $\tht(i,\ep)=(-1)^i\ep$ is a chequer function.  It follows that we
 have a universe as claimed.
\end{proof}

\begin{proposition}\label{prop-dowker-b}
 Let $A$ be any universe whose underlying $\Dl$-set is $DT_0(n)$; then
 there is a unique pair $(\sg,\phi)$ such that $A=DT(\sg,\phi)$.
\end{proposition}
\begin{proof}
 Each $\rho$-orbit must consist of two distinct $\rho^2$-orbits, and
 each $\rho^2$-orbit contains a unique point whose second component is
 $+1$.  Thus, for each $i\in\Z/(2n)$ there exists a unique element
 $\sg(i)\in\Z/(2n)\sm\{i\}$ such that $(i,1)$ and $(\sg(i),1)$
 generate the same $\rho$-orbit.  It is clear from this
 characterisation that $\sg$ is an involution without fixed points.
 Note that $(i,1)$ and $(\sg(i),1)$ cannot be linked by $\rho^2$, so
 they must be linked by $\rho^{\pm 1}$.  If
 $(\sg(i),1)=\rho^{\pm 1}(i,1)$ then we find that
 $\pi^{i-\sg(i)}\rho^{\pm 1}$ stabilises $(i,1)$, so we must have
 $c(\pi^{i-\sg(i)}\rho^{\pm 1})=1$, so $\sg(i)=1-i\pmod{2}$.  Now put
 $\phi(i)=1$ if $(\sg(i),1)=\rho(i,1)$, and $\phi(i)=-1$ if
 $(\sg(i),1)=\rho^{-1}(i,1)$.  Let $\rho_0$ denote the map defined
 using $\sg$ and $\phi$ as in Definition~\ref{defn-dowker}.  This has
 $\rho_0^2=\rho^2$, and $\rho_0$ has the same orbits as $\rho$, and
 $\rho_0$ agrees with $\rho$ on at least one point in each orbit; it
 follows easily that $\rho=\rho_0$.
\end{proof}

\begin{corollary}
 Let $A$ be a universe with only one component.  Then
 $A\simeq D(\sg,\phi)$ for some $\sg$ and $\phi$.
\end{corollary}
\begin{proof}
 Using Remark~\ref{rem-dihedral} we see that $A$ is
 $\Dl$-equivariantly isomorphic to $DT_0(n)$ for some $n$.  Given
 this, the claim follows easily from Proposition~\ref{prop-dowker-b}. 
\end{proof}

\begin{remark}
 Note that $DT(\sg,-\phi)$ is the mirror image of $DT(\sg,\phi)$, so
 it is spherical if and only if $DT(\sg,\phi)$ is spherical.
\end{remark}

\begin{theorem}
 Let $\sg$ be an involution on $\Z/(2n)$ with $\sg(i)=1-i\pmod{2}$,
 and put 
 \[ \Phi = \{\phi\st DT(\sg,\phi) \text{ is spherical }\}. \]
 Then, under mild irreducibility conditions, we have either
 $\Phi=\emptyset$ or $\Phi=\{\phi,-\phi\}$ for some $\phi$.
\end{theorem}
\begin{proof}
 I think that this is an accurate reformulation of a result proved by
 Dowker and Thistlethwaite.  However, I have not yet digested the proof.
\end{proof}

\end{document}
